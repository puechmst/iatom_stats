
\documentclass[main.tex]{subfiles}
\begin{document}
\section{Estimation}
\begin{frame}[fragile]
    \frametitle{Introductory example}
\begin{block}{Delays dataset}
   \begin{itemize}
    \item<+-> Download the file "delays.csv" and open it in PowerBI. Use the "Table" pane and find the maximum and minimum values of the
    third column. 
    \item<+-> Negative delays are not of interest for us, not delays higher than 10h, that are outliers. Create a new table with the command:
    \begin{verbatim}
        pos_delays = FILTER(delays,delays[Column3]>=0 && delays[Column3]<=36000)
    \end{verbatim}
    \item<+-> Using an "R" visual, plot the histogram of the delays in the new table.
\end{block}
\end{frame}
\begin{frame}
    \frametitle{Introductory example}
\begin{block}{Modeling the delays}
   \begin{itemize}
    \item<+-> Compute the inverse of the average of the delays. 
   \end{itemize} 
\end{block}
    

\end{frame}

\end{document}