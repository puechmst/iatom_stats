\documentclass[main.tex]{subfiles}
\begin{document}
\section{Estimation}
\begin{frame}[fragile]
    \frametitle{Introductory example}
\begin{block}{Delays dataset}
   \begin{itemize}
    \item<+-> Download the file "delays.csv" and open it in PowerBI. Use the "Table" pane and find the maximum and minimum values of the
    third column. 
    \item<+-> Negative delays are not of interest for us, nor delays higher than 3h, that are outliers. Create a new table with the command:
    {\footnotesize 
    \begin{verbatim}
pos_delays = 
FILTER(delays,delays[Column3]>=0 && delays[Column3]<=10000)
    \end{verbatim}}
    \item<+-> Using an "R" visual, plot the histogram of the delays in the new table with the option \texttt{freq=FALSE} to display probabilities.
    Since PowerBI removes duplicate values, be sure to include all columns in the plot.
\end{itemize}
\end{block}
\end{frame}
\begin{frame}
    \frametitle{Introductory example}
\begin{block}{Modeling the delays}
   \begin{itemize}
    \item<+-> Compute the inverse of the average of the delays.
    \item<+-> Use an R script to generate an exponentially distributed sample of size 100000 (use the \texttt{rexp} command) with rate the above value.
    \item<+-> Compare the histogram of the values with the one coming from the delays dataset. 
    \item<+-> Do you think that assuming an exponential distribution for the delays is reasonable ?
   \end{itemize} 
\end{block}
\end{frame}
\begin{frame}
    \frametitle{Estimation}
\begin{block}{Population vs sample}
    \begin{itemize}
        \item<+-> In the previous example, the dataset of delays is a sample, i.e. a set of \textbf{observations}.
        \item<+-> The associated population is an hypothetical measure space describing all possible delays.
    \end{itemize}
\end{block}
\begin{block}{Statistical model}
   \begin{itemize}
    \item<+-> An statistical model is a triple $\left( \Omega, \mathcal{T}, \mathcal{P} \right)$ where $\Omega$ is a sample space,
    $\mathcal{T}$ a $\sigma$-algebra on $\Omega$ and $\mathcal{P}$ a set of probabilities on $\mathcal{T}.$
    \item<+-> The population associated with the studied sample hopefully belongs to $\mathcal{P}$\dots
   \end{itemize} 
\end{block}
\end{frame}
\begin{frame}
    \frametitle{Estimation}
\begin{block}{Kinds of models}
\begin{itemize}
    \item<+-> If $\mathcal{P}$ can be fully described by a finite number of parameters, the model is said to be \textbf{parametric}. As an example, 
    the set of exponential distributions is parameterized by the rate.
    \item<+-> If $\mathcal{P}$ can be partly described by a finite number of parameters, the model is said to be \textbf{semi-parametric}.
\end{itemize}    
\end{block}
    

\end{frame}
\end{document}