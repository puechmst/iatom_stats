\section{Probability theory}

%1
\begin{frame}
    \frametitle{A bit of history}
\begin{block}{Games of chance}
    \begin{itemize}
        \item<+-> According to popular belief, knights returning from the Crusades introduced a dice game named "Hazard."
         Although the rules were complicated, the game attracted many players.
        \item<+-> The mathematician Gerolamo Cardano, in the middle of the 16th century, starts to investigate the odds of winning
        in a game of chance. He gives the first definition of the probability of an event as the ratio of
        the number of favorable outcomes to the total number of outcomes.
        \item<+-> One century later, Pierre de Fermat and Blaise Pascal lay the first theory of probability when
        answering a question from Antoine Gombaud, a famous Parisian gambler.
    \end{itemize}
\end{block}
\end{frame}
\begin{frame}
    \frametitle{A bit of history}
\begin{block}{A modern view}
    \begin{itemize}
        \item<+-> In 1933, the Russian mathematician Andrey Kolmogorov gives the first axiomatic description of probability theory.
        \item<+-> This is the approach we will take in the sequel.
    \end{itemize}
\end{block}    
\begin{block}{Challenges}
    \begin{itemize}
       \item<+-> To fully understand Kolmogorov's contribution, we must consider cases in which counting the number of favorable outcomes is insufficient.
       \item<+-> For example, take a situation where the measure of interest is a real number, such as a current or a pressure.
       \item<+-> The value itself is not discrete; only the number of times it falls within a given interval can be observed.
       \item<+-> The Fermat-Pascal approach must then be extended to these cases.
    \end{itemize}
\end{block}    
\end{frame}