\documentclass[main.tex]{subfiles}
\usepackage{tikz}
\usetikzlibrary{arrows.meta, positioning}
\usepackage{graphicx}

\begin{document}
\begin{frame}[fragile]
    \frametitle{Exponential distribution}
    \begin{itemize}
        \item<+-> This exercise is an illustration of the method of moments.
        \item<+-> Using PowerBI and R, generate a population from an exponential distribution with rate 2:
\begin{verbatim}
sim_exp <- data.frame(n=1:100000,x=rexp(100000,rate=2))
\end{verbatim}
        \item<+-> Compute the mean $m$ and the standard deviation $\sigma$ (square roort of the variance) on $1000$ samples is size $1000$.
        \item<+-> Estimate the number of times $\lvert m - 0.5 \rvert$ (resp. $\lvert \sigma - 0.5 \rvert$) is greater than $0.026.$
        \item<+-> In your opinion, which estimator of the inverse of the rate is the best one ?
    \end{itemize}
\end{frame}

\begin{frame}
    \frametitle{Confidence intervals}
    \begin{itemize}
        \item<+-> A consistent estimator $\hat{\theta}$ will fall within a given interval around the true value $\theta$ with a probability close to 1
        if the sample size is large enough.
        \item<+-> In practice, the sample size is fixed, so that one wants to find $c$ such that:
        \begin{equation}
            P\left[ \lvert \hat{\theta} - \theta \rvert > c \right] = \alpha,
        \end{equation}
        where $\alpha \in [0,1]$ is a fixed value.
        \item<+-> The interval:
        \[
        \left[ \hat{\theta}-c, \hat{\theta} + c \right]
        \]
        contains $\theta$ with probability $1- \alpha.$
        \item<+-> $1-\alpha$ is called the confidence level.
    \end{itemize}
\end{frame}

\begin{frame}
    \frametitle{Confidence intervals}
    \begin{itemize}
        \item<+-> Generally speaking, a confidence interval for $\theta$ at level $1-\alpha$ is a pair of estimators $\hat{\theta}_1, \hat{\theta}_2$ 
        such that:
        \[
        P\left[\theta \in  \left[ \hat{\theta}_1, \hat{\theta}_2\right] \right] = 1-\alpha.
        \]
        \item<+-> A classical situation is when $\hat{\theta}$ is the sample mean:
        \[
        \hat{\theta} = \bar{X} = \frac{1}{n} \sum_{i=1}^n X_i.
        \]
    \end{itemize}
\end{frame}

\begin{frame}
    \frametitle{Confidence intervals}
    \begin{itemize}
        \item<+-> Generally speaking, a confidence interval for $\theta$ at level $1-\alpha$ is a pair of estimators $\hat{\theta}_1, \hat{\theta}_2$ 
        such that:
        \[
        P\left[\theta \in  \left[ \hat{\theta}_1, \hat{\theta}_2\right] \right] = 1-\alpha.
        \]
        \item<+-> A classical situation is when $\hat{\theta}$ is the sample mean:
        \[
        \hat{\theta} = \bar{X} = \frac{1}{n} \sum_{i=1}^n X_i.
        \]
    \end{itemize}
\end{frame}

\begin{frame}
    \frametitle{Using the TCL}
\begin{itemize}
    \item<+-> When the sample mean is an unbiased estimator of $\theta$, then, in the limit of large samples:
    \begin{equation}
        Z = \frac{\sqrt{n}\left( \bar{X} - \theta \right)}{\sigma}
    \end{equation}
    is $\mathcal{N}(0,1)$-distributed.
    \item<+-> For a given confidence level $1-\alpha$, one can find $c$ such that:
    \begin{equation}
        P\left[ \lvert Z \rvert \leq c \right] = 1-\alpha.
    \end{equation}
    \item<+-> A confidence interval for $\theta$ is then:
    \begin{equation}
        \left[ \bar{X} - \frac{\sigma c}{\sqrt{n}}, \bar{X} + \frac{\sigma c}{\sqrt{n}}\right]
    \end{equation}
    \item<+-> The approximation is correct provided $n \geq 30.$
\end{itemize}
\end{frame}
\end{document}
