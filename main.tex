\documentclass[10pt,xcolor=svgnames,handout]{beamer} %Beamer
\usepackage{palatino} %font type
\usefonttheme{metropolis} %Type of slides
\usefonttheme[onlymath]{serif} %font type Mathematical expressions
\usetheme[progressbar=frametitle,titleformat frame=smallcaps,numbering=counter]{metropolis} %This adds a bar at the beginning of each section.
\useoutertheme[subsection=false]{miniframes} %Circles in the top of each frame, showing the slide of each section you are at

\usepackage{appendixnumberbeamer} %enumerate each slide without counting the appendix
\setbeamercolor{progress bar}{fg=Maroon!70!Coral} %These are the colours of the progress bar. Notice that the names used are the svgnames
\setbeamercolor{title separator}{fg=DarkSalmon} %This is the line colour in the title slide
\setbeamercolor{structure}{fg=black} %Colour of the text of structure, numbers, items, blah. Not the big text.
\setbeamercolor{normal text}{fg=black!87} %Colour of normal text
\setbeamercolor{alerted text}{fg=DarkRed!60!Gainsboro} %Color of the alert box
\setbeamercolor{example text}{fg=Maroon!70!Coral} %Colour of the Example block text


\setbeamercolor{palette primary}{bg=NavyBlue!50!DarkOliveGreen, fg=white} %These are the colours of the background. Being this the main combination and so one. 
\setbeamercolor{palette secondary}{bg=NavyBlue!50!DarkOliveGreen, fg=white}
\setbeamercolor{palette tertiary}{bg=NavyBlue!40!Black, fg= white}
\setbeamercolor{section in toc}{fg=NavyBlue!40!Black} %Color of the text in the table of contents (toc)

%These next packages are the useful for Physics in general, you can add the extras here. 
\usepackage{amsmath,amssymb}
\usepackage{slashed}
\usepackage{cite}
\usepackage{relsize}
\usepackage{caption}
\usepackage{subcaption}
\usepackage{multicol}
\usepackage{booktabs}
\usepackage[scale=2]{ccicons}
\usepackage{pgfplots}
\usepgfplotslibrary{dateplot}
\usepackage{geometry}
\usepackage{xspace}
\newcommand{\themename}{\textbf{\textsc{bluetemp}\xspace}}%metropolis}}\xspace}
\DeclareMathOperator{\sinc}{sinc}
\graphicspath{{./images/}}
\title{Mathematics for signal processing}
\author[Name]{Stéphane Puechmorel \inst{$\dagger$}}
\subtitle{An introduction}
\institute[uni]{\inst{$\dagger$} ENAC Dept. SINA \\ email: stephane.puechmorel@enac.fr \\ phone: 0562259503}
\date{\today} %Here you can change the date
\titlegraphic{\vspace{-0.5cm}\hfill\includegraphics[scale=0.3]{images/ENAC-Orange.png}} %You can modify the location of the logo by changing the command \vspace{}. 

\definecolor{almond}{rgb}{0.94, 0.87, 0.8}
\definecolor{bubbles}{rgb}{0.91, 1.0, 1.0}
\newcommand{\R}{\ensuremath{\mathbb{R}}}
\newcommand{\C}{\ensuremath{\mathbb{C}}}
\newcommand{\N}{\ensuremath{\mathbb{N}}}

\begin{document}
{
\setbeamercolor{background canvas}{bg=NavyBlue!50!DarkOliveGreen, fg=white}
\setbeamercolor{normal text}{fg=white}
\maketitle
}%This is the colour of the first slide. bg= background and fg=foreground

\metroset{titleformat frame=smallcaps} %This changes the titles for small caps

\begin{frame}{Outline}
  \setbeamertemplate{section in toc}[sections numbered] %This is numbering the sections
  \tableofcontents[hideallsubsections] %You can comment this line if you want to show the subsections in the table of contents
\end{frame}




\section{Introduction}

% \begin{frame}{Objectives}
% %\underline{\textsc{Some text:}}
% % \begin{small}
% % This is some small Text. 
% % \end{small}

% % \metroset{block=fill}
% % \begin{exampleblock}{\textsc{Example block}}
% % \begin{itemize}
% %     \item You know how to do itemize
% %     \item Also here
% % \end{itemize}
% % \end{exampleblock}
% \end{frame}

%1
\begin{frame}[fragile]{Introduction to statistics and data science}
\begin{block}{Why this course ?}
\begin{itemize}
    \item<+-> Data are omnipresent in engineering and business intelligence.
    \item<+-> Understanding them can often be challenging.
    \item<+-> Statistics provides the data analyst with a rigorous framework.
\end{itemize}
\end{block}
\begin{block}{Course contents:}
\begin{itemize}
    \item<+-> Concepts from probability theory.
    \item<+-> Visual data exploration.
    \item<+-> Inferential statistics.
    \item<+-> Hypothesis testing.
\end{itemize}
\end{block}
\end{frame}

%2
\begin{frame}[fragile]{Work organization}
\begin{itemize}
    \item<+-> All lectures will include course material and exercises.
    \item<+-> The notions are illustrated using real data and statistical software.
    \item<+->Should you encounter any difficulties, please do not hesitate to inquire.
    \item<+->Personal work is essential and must be completed regularly.
\end{itemize}
\end{frame}

%3
\begin{frame}[fragile]{Software}
\begin{itemize}
    \item<+-> Two software products will be utilized during the course:
    \begin{itemize}
        \item<+-> R, that is widely recognized as the prevailing standard within the statistical community.
        \item<+-> PowerBI, a tool that is being used with increasing frequency for the collection, processing, and visualization of data. It integrates almost seamlessly with R. 
    \end{itemize}
    \item<+-> Advanced concepts will not be covered. 
    \item<+-> Feel free to install both on your personal computer. Power BI Desktop cannot share dashboards
     unless you have a paid subscription, but this is not a significant issue when learning how to use the tool. 
    \end{itemize}
\end{frame}
\include{integration}
\include{fourier}
\include{complex}
\include{distribution}


% \begin{frame}[standout]{This is other type of slide}
% There is some text here.
% And an equation with number:
% \begin{equation}
%     E^{2} = m^{2} + p^{2}
% \end{equation}
% \end{frame}


% {\setbeamercolor{palette primary}{fg=black, bg=orange!30} %You can change the colours
% \begin{frame}[standout]
%   Thank you! And thank to yourself because you did all the job. 
% \end{frame}
% }
\appendix

% \begin{frame}{Back up}
%     These slides won't appear in the table of contents and will not be counted as the total slides.
% \end{frame}

\end{document}
